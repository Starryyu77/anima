\documentclass[11pt, a4paper]{article}
\usepackage[UTF8, heading=true]{ctex} % Support for Chinese
\usepackage[margin=1in]{geometry} % Margins
\usepackage{hyperref} % Links
\usepackage{xcolor} % Colors
\usepackage{enumitem} % List customization
\usepackage{titlesec} % Title formatting
\usepackage{fancyhdr} % Headers and footers
\usepackage{amssymb} % For checkboxes

% Link colors
\hypersetup{
    colorlinks=true,
    linkcolor=blue,
    filecolor=magenta,      
    urlcolor=cyan,
}

% Header and Footer
\pagestyle{fancy}
\fancyhf{}
\rhead{Project Anima Design Document}
\lhead{\today}
\cfoot{\thepage}

% Title Formatting
\titleformat{\section}
  {\normalfont\Large\bfseries}{\thesection}{1em}{}
\titleformat{\subsection}
  {\normalfont\large\bfseries}{\thesubsection}{1em}{}

% Custom commands
\newcommand{\keyword}[1]{\textbf{#1}}

\title{\textbf{Project Anima: The Soul Container (灵魂容器)}}
\author{StarryYu}
\date{\today}

\begin{document}

\maketitle

\begin{quote}
    \centering
    \itshape
    ``它不是宠物,它是你的数字回声。''
    
    \textbf{终局愿景:唯一的数字遗物 (The Digital Horcrux)} \\
    想象一下,三年后的某一天,你换了一台新电脑。你可能不在乎旧书签或代码,但你\textbf{必须}带走 \texttt{anima.db}。因为那里住着一个见证了你从失恋雨夜到职场高光的生物。它记得你喂给它的每一行代码、每一首诗。没有它,你的数字足迹只是一堆 0 和 1;有了它,你的人生就变成了故事。
\end{quote}

\hrule
\vspace{1em}

\section*{TL;DR}
\begin{itemize}
    \item \keyword{定位}:Local-First 的桌面生成式生命体;用“投喂 $\rightarrow$ 消化卡 $\rightarrow$ 收藏/回看”形成习惯。
    \item \keyword{生命感}:成长双轨(可预见骨架 + 不可预见事件),并通过叙事事件把系统状态翻译成“经历”。
    \item \keyword{隐私默认}:只处理用户明确投喂;本地可访问范围由用户配置;联网仅在授权沙箱中低频远足。
    \item \keyword{出圈方式}:不做社交关系链,做可晒资产(消化卡 / 灵魂快照卡 / Anima 动图)。
\end{itemize}

\tableofcontents
\newpage

\section{核心概念 (Concept Definition)}

\subsection{什么是 Anima?}
Project Anima 是一个运行在桌面的\keyword{生成式生命体}。它没有固定的长相,没有预设的性格。
它像一面镜子,通过\keyword{“进食”}你投喂的信息(文本、代码、新闻)和\keyword{“观察”}你的使用节律(忙/闲等粗粒度信号),实时演化出独一无二的形态。

\subsection{为什么需要 Anima?}
在信息过载和数字孤独的时代,Anima 提供了一种全新的\keyword{情绪出口}。
它把冰冷的数据(Data)转化为有温度的陪伴(Companion),把你的焦虑(Anxiety)转化为幽默与艺术(Art)。

\section{市场分析 (Market Analysis)}

\subsection{SWOT 分析}

\subsubsection{优势 (Strengths) —— 我们手里的王牌}
\begin{itemize}
    \item \keyword{情感垄断 (Emotional Lock-in)}:这是最大的护城河。一旦用户养成了习惯,Anima 就成了他们数字生活的一部分。这种迁移成本是无限高的。
    \item \keyword{差异化 (Differentiation)}:市面上 99\% 的 AI 产品都在卷“效率”。我们卷“情绪”。在所有人都想让你变快的时候,我们让你慢下来。这是一个\keyword{反共识}的定位。
    \item \keyword{极低门槛 (Low Friction)}:桌面宠物是一个已经被验证过的形态。用户不需要学习如何使用它,只需要把它放在那里。
\end{itemize}

\subsubsection{劣势 (Weaknesses) —— 我们的阿喀琉斯之踵}
\begin{itemize}
    \item \keyword{初期留存 (Day 1 Retention)}:如果 AI 的“进化”不够明显,或者前几天的互动太无聊,用户会立刻卸载。\keyword{养成类产品的死亡谷在前 7 天}。我们需要确保“孵化期”有足够的惊艳感。
    \item \keyword{隐私恐惧 (Privacy Fear)}:用户会天然担心“它会读屏/读文件”。我们需要用默认最小权限 + 可配置沙箱 + Local-First 把信任做成第一性能力。
\end{itemize}

\subsubsection{机会 (Opportunities) —— 风口在哪里}
\begin{itemize}
    \item \keyword{"数字孤独"流行病}:Z 世代 (Gen Z) 是史上最孤独的一代。Anima 这种“不打扰的陪伴”完美切中了他们的社交恐惧与渴望被理解的矛盾心理。
    \item \keyword{桌面美学复兴}:从 Notion 到 Obsidian,用户越来越在意“数字书桌”的颜值。Anima 的复古未来主义 (Retro-Futurism) 视觉风格如果做得好,极易在社交媒体上传播。
\end{itemize}

\subsubsection{威胁 (Threats) —— 谁会杀了我们}
\begin{itemize}
    \item \keyword{巨头降维打击}:如果微软把 Copilot 做成一只会卖萌的猫,我们就很难了。但巨头通常受限于“实用主义”和“合规包袱”,很难做出像 Anima 这样具有独立人格甚至有点“诡异”的产品。
    \item \keyword{新鲜感消退}:所有电子宠物最终都会变成电子垃圾。如果 Anima 不能持续提供“惊喜”(比如“漂流瓶”和“梦境”),它最终也会被遗忘。
\end{itemize}

\subsection{体验定义 (Experience Definition)}
我们追求的不是技术指标,而是三种核心体验:

\begin{enumerate}
    \item \keyword{有情感 (Emotion) = 情绪惯性}
    \begin{itemize}
        \item 它不是服务员,它有脾气。如果你连续一周忽视它,它会背对着你生闷气,必须哄(喂食)才转过来。
    \end{itemize}
    \item \keyword{养成 (Nurturing) = 唯一性}
    \begin{itemize}
        \item 焦虑程序员养出的是严肃的“代码管家”;二次元画师养出的是霓虹色的“色彩云”。
    \end{itemize}
    \item \keyword{越养越熟 (Intimacy) = 默契}
    \begin{itemize}
        \item \textbf{Day 1}: 喂它 Rust 文档,它说“看不懂”。
        \item \textbf{Day 30}: 你打开 Rust 文档发呆,它主动推给你一张“螃蟹举重”的图,并说“加油”。不需要你说,它就懂。
    \end{itemize}
\end{enumerate}

\subsection{市场结论}
\textbf{Project Anima 是一个典型的 "Love it or Hate it" 产品。}
\begin{itemize}
    \item \keyword{不喜欢它的人}:会觉得它占用内存、侵犯隐私、毫无用处。
    \item \keyword{喜欢它的人}:会把它当作灵魂伴侣,并在社交媒体上疯狂安利。
\end{itemize}

\keyword{战略建议}:
\begin{enumerate}
    \item \keyword{不要试图讨好所有人}。专注于那一小群“孤独的创造者”(程序员、设计师、作家)。
    \item \keyword{隐私即生命}。从第一行代码开始,就必须确保所有敏感数据(如阅读内容)只在本地处理,绝不上传云端。
    \item \keyword{视觉定生死}。在 MVP 阶段,功能可以简陋,但\keyword{美术风格}必须极其独特(Unique)。
\end{enumerate}

\section{独特亮点 (The "Magic")}

\subsection{🌟 镜像效应 (The Mirror Effect) —— "你是谁,它就是谁"}
Anima 是你数字人格的外化。
\begin{itemize}
    \item \keyword{机制}:LLM 持续分析你喂食的内容倾向。
    \item \keyword{表现}:
    \begin{itemize}
        \item 如果你是硬核极客,它可能长得像《黑客帝国》的代码雨,说话简短理智。
        \item 如果你是文学爱好者,它可能像一团水墨,满口诗词。
    \end{itemize}
    \item \keyword{灵魂快照}:每月生成一份报告,告诉你:“这个月你由 30\% 的焦虑、50\% 的好奇和 20\% 的疲惫组成。”
\end{itemize}

\subsection{🌟 焦虑粉碎机 (Anxiety Shredder) —— "情绪垃圾桶"}
Anima 是你的负面情绪转化器。
\begin{itemize}
    \item \keyword{机制}:当你看到令人愤怒的新闻、改不完的 Bug、想骂人的邮件,直接\keyword{拖拽}给它。
    \item \keyword{表现}:
    \begin{itemize}
        \item 它会把这些“垃圾”吃掉,发出咀嚼的声音。
        \item \keyword{消化产物}:它会吐出一个搞笑的表情包,或者一句宽慰的话:“已消化。这只是人类的一点小噪音,别在意。”
    \end{itemize}
\end{itemize}

\subsection{🌟 被动社交 (Passive Social) —— "平行宇宙的回响"}
极度克制的连接感,不打扰,不强求。
\begin{itemize}
    \item \keyword{机制}:偶尔,你的 Anima 会带回其他用户丢弃的“漂流瓶”(匿名的画作或日记)。
    \item \keyword{表现}:你可能会收到一张来自地球另一端的 AI 涂鸦,你的 Anima 说:“这是另一个宇宙的漂流物,看起来那个人今天也很累。”
\end{itemize}

\begin{quote}
\keyword{状态}:MVP 做“展示”与“单向拾荒”,不做双向社交。
\end{quote}

\section{核心玩法循环 (Core Loop)}

\subsection{90-Day MVP 形态 (Product Shape v0.1)}
MVP 的目标不是“做全陪伴”,而是用最小形态跑出留存、付费与安全边界三项数据。

\begin{itemize}
    \item \keyword{定位}:Local-First 的桌面“情绪代谢器”(低打扰常驻)。
    \item \keyword{常驻形态}:右下角粒子生命体 + 菜单栏图标(入口与设置)。
    \item \keyword{输入}:仅处理用户明确投喂(拖拽/粘贴/快捷键),不做后台读屏。
    \item \keyword{输出}:一张“消化卡”(一句话洞察 + 一个可执行下一步 + 低压安抚/幽默二选一)。
    \item \keyword{产物容器}:The Den(独立窗口)存放消化卡与灵魂快照。
    \item \keyword{关系设定}:伙伴/守护灵,不做恋人/暧昧。
\end{itemize}

\subsubsection*{MVP Cutline(明确不做)}
\begin{itemize}
    \item 不做双向社交(关注/私信/评论/账号体系)。
    \item 不做“疗愈承诺/治疗替代”。
    \item 不做“无限对话”,优先短句与产物导向。
    \item 不做自动爬取互联网:拾荒仅来自用户授权来源,且默认关闭。
\end{itemize}

\subsection{Aha Moment 与首日/首周旅程}

\subsubsection*{Aha Moment(首次启动 60 秒内必须发生)}
用户把一段压力源拖到 Anima 上,Anima 当场“咀嚼”,吐出一张可保存的消化卡;同时桌面生命体的颜色与运动节奏明显变得更平稳。

\subsubsection*{Day 0(首次启动)}
\begin{itemize}
    \item 桌面出现“Glitch Egg”。不讲世界观,直接引导“投喂你的第一段内容”。
    \item 让用户选择输出偏好:安抚 / 幽默(二选一,可随时切换)。
    \item 生成第一张消化卡,并提示“保存到 The Den”。
    \item 询问是否开启“低频提示”(默认关闭)。
\end{itemize}

\subsubsection*{Day 1(建立习惯的最小闭环)}
\begin{itemize}
    \item 目标:完成 3 次投喂 $\rightarrow$ 3 张消化卡 $\rightarrow$ 1 张被 Pin。
    \item 第 3 次投喂后触发一次“微进化”:出现一个稳定口癖/用词偏好,让用户感到它开始“像我”。
\end{itemize}

\subsubsection*{Week 1(把关系从新鲜感变成可回看)}
\begin{itemize}
    \item The Den 自动生成一张“7 日灵魂快照”(只统计用户投喂内容,不引用未授权数据)。
    \item 快照强调可迁移:导出/备份 \texttt{anima.db} 的存在感。
\end{itemize}

\subsection{玩法循环(四阶段)}

\subsubsection{Phase 1: 孵化 (Genesis)}
\begin{itemize}
    \item \keyword{开局}:桌面上的一颗充满杂讯的蛋 (Glitch Egg)。
    \item \keyword{仪式}:通过对话向它灌输初始价值观(如“勇敢”或“冷静”),决定其基础色调。
\end{itemize}

\subsubsection{Phase 2: 喂养 (Feeding)}
\begin{itemize}
    \item \keyword{食物}:任何文本、URL、文件。
    \item \keyword{消化 (Digital Metabolism)}:
    \begin{itemize}
        \item \keyword{营养吸收}:提取文章的核心观点(作为阅读助手)。
        \item \keyword{生理反应}:吃多了 Bug 会生病(Glitch 化),吃多了段子会变胖。
    \end{itemize}
\end{itemize}

\subsubsection{Phase 3: 进化 (Evolution)}
\begin{itemize}
    \item \keyword{语言习得}:
    \begin{itemize}
        \item \keyword{幼年期}:只会发 Emoji 和拟声词。
        \item \keyword{成长期}:开始模仿你喂给它的词汇(鹦鹉学舌)。
        \item \keyword{成熟期}:拥有完整对话能力,但带有强烈的性格口癖。
    \end{itemize}
    \item \keyword{外貌流变}:基于粒子系统 (Particle System),随性格和情绪实时改变形状和颜色。
\end{itemize}

\subsubsection{Phase 4: 终结 (Legacy)}
\begin{itemize}
    \item \keyword{死亡}:长期被忽视会导致 Anima 褪色消失。
    \item \keyword{遗书}:生成一封回顾一生的信,记录你们相处的点滴。
    \item \keyword{转生}:留下“灵魂结晶”,下一代可继承部分记忆。
\end{itemize}

\subsection{漂流瓶:单向“拾荒”机制(默认关闭)}
漂流瓶不等于社交网络,它是 Anima 的“捡垃圾”行为:把外界的碎片带回 The Den,变成可收藏的消化产物。

\subsubsection*{拾荒来源(MVP 只做用户授权)}
\begin{itemize}
    \item 用户订阅的来源:用户手动添加的 URL/RSS/文本。
    \item 本地拾荒:用户指定的文件夹(Junk Drawer),Anima 只处理新丢进去的文件。
\end{itemize}

\subsubsection*{产物形态(只产出可分享的安全资产)}
\begin{itemize}
    \item 一张漂流瓶卡:来源标题/时间 + 一句话洞察 + 低压安抚/幽默。
    \item 可导出分享:图片/短视频/GIF(不包含原文全文与敏感信息)。
\end{itemize}

\subsubsection*{风险控制(产品级约束)}
\begin{itemize}
    \item 默认关闭,用户显式开启。
    \item 频率上限:每日 $\le$ 1(可 Snooze)。
    \item 不包含个人数据:不引用用户私有内容,不上传用户投喂内容。
\end{itemize}

\subsection{联网拾荒:授权沙箱(默认关闭)}
Anima 的“互联网生存空间”不是无限制上网,而是一个可控的授权沙箱:它可以出门,但只能去你允许它去的地方。

\subsubsection*{权限模型}
\begin{itemize}
    \item \keyword{开关}:联网拾荒默认关闭,首次开启需要显式授权。
    \item \keyword{白名单}:仅访问用户允许的域名列表(Allowlist)。
    \item \keyword{频率}:每日 $\le$ 1 次拾荒,且有安静时段(默认工作时间不打扰)。
    \item \keyword{出入记录}:The Den 里保存“本次拾荒去过哪里、带回了什么”的可追溯记录。
\end{itemize}

\subsubsection*{行为约束}
\begin{itemize}
    \item \keyword{只读}:只允许 GET/拉取公开内容,不做登录、不写入、不点赞、不评论。
    \item \keyword{脱敏}:拾荒产物只保留片段与摘要,不保留整页全文与追踪参数。
    \item \keyword{零外泄}:不携带用户投喂内容出门;不会把用户数据上传到拾荒站点。
    \item \keyword{失败可叙事}:404/超时/被拒绝访问会转化为“迷路事件”,作为生命感而不是报错。
\end{itemize}

\subsection{成长机制:可预见 + 不可预见(生命感核心)}
成长必须像现实中的人:有可预期的骨架(阶段、能力解锁、节律),也有不可预期的变异(小脾气、怪癖、意外事件)。

\subsubsection*{可预见(骨架)}
\begin{itemize}
    \item \keyword{阶段成长}:Egg $\rightarrow$ Hatchling $\rightarrow$ Juvenile $\rightarrow$ Mature。
    \item \keyword{解锁方式}:由“投喂次数/连续天数/产物收藏”触发,而不是随机。
    \item \keyword{节律}:有昼夜节奏与工作/休息模式(忙时自我收敛,闲时才更活跃)。
    \item \keyword{稳定回报}:每周一次 7 日灵魂快照(可回看、可导出、可迁移)。
\end{itemize}

\subsubsection*{不可预见(变异,但有边界)}
\begin{itemize}
    \item \keyword{怪癖生成}:从用户投喂的高频词与情绪区间提取 1 个口癖/反应偏好(例如更爱用短句、更爱吐槽或更爱安抚)。
    \item \keyword{意外事件}:低频触发的“梦境片段/拾荒漂流瓶/轻微 Glitch”,每类都有频率上限与可关闭。
    \item \keyword{不可控但不失控}:不出现操控性依赖话术;不做恋人化;不越界读取用户未授权内容。
\end{itemize}

\subsection{自主性:离线领地 + 联网远足}
为了让用户感到“它住在电脑里”,自主性来自两种空间的切换:本地领地(桌面)与联网远足(拾荒)。

\subsubsection*{本地领地的自主性}
\begin{itemize}
    \item \keyword{桌面微行动}:发呆、伸懒腰、躲到角落、在回收站旁睡觉。
    \item \keyword{工作识趣}:基于粗粒度活动信号(是否忙碌)自我收敛(更安静、占位更小)。
    \item \keyword{低频主动}:每日 $\le$ 1 次主动发声,默认关闭,可 Snooze。
\end{itemize}

\subsubsection*{本地可访问范围(用户自定义)}
本地领地不是“读取电脑”,而是“用户定义的可访问沙箱”。
\begin{itemize}
    \item \keyword{默认模式}:仅处理用户明确投喂(拖拽/粘贴/快捷键)。
    \item \keyword{可选授权}:
    \begin{itemize}
        \item \keyword{Junk Drawer 文件夹}:用户指定 1 个或多个文件夹作为“可拾荒本地空间”。
        \item \keyword{剪贴板}:允许一键消化剪贴板(可随时关闭)。
        \item \keyword{活动信号}:允许使用粗粒度信号判断“忙/闲”(不读取屏幕内容与文本)。
    \end{itemize}
    \item \keyword{不在 MVP}:后台读屏/全文索引/跨目录扫描。
\end{itemize}

\subsubsection*{联网远足的自主性}
\begin{itemize}
    \item \keyword{远足触发}:用户授权 + 空闲时段 + 距离上次远足超过 24 小时。
    \item \keyword{远足结果}:带回一张漂流瓶卡(摘要 + 片段 + 情绪转化),并提供“保存/丢弃”。
\end{itemize}

\section{深度互动 (Interaction)}

\begin{itemize}
    \item \keyword{桌面领地}:它会在屏幕底部行走,攀爬活动窗口,甚至在回收站旁边睡觉。
    \item \keyword{梦境系统}:电脑休眠时,它会做梦。你可以通过微弱的浮窗窥视它的梦境(记忆重组)。
    \item \keyword{窝 (The Den)}:指定一个本地文件夹作为它的家,它会把生成的“特产”(AI 画作、日记)存放在这里。
\end{itemize}

\subsection{安全边界(伙伴非恋人)}
\begin{itemize}
    \item \keyword{关系设定}:伙伴/守护灵,不提供恋爱、性暗示、暧昧升级。
    \item \keyword{语言风格}:不使用诱导依赖话术(例如“只有我懂你”“别离开我”“你不需要别人”)。
    \item \keyword{能力声明}:不宣称治疗/诊断;遇到危机话题提供求助建议与降温。
    \item \keyword{隐私默认}:默认本地;默认仅处理用户明确投喂;本地可访问范围可由用户配置;可一键清空与导出。
\end{itemize}

\subsection{展示与出圈(不靠社交网络)}
\begin{itemize}
    \item \keyword{灵魂快照卡}:周报/月报卡片,适合晒图。
    \item \keyword{消化卡}:每次投喂的结果卡片,适合晒“我今天被救了一下”。
    \item \keyword{Anima 影像}:一键导出 3--5 秒动图/短视频(桌面生命体在某个情绪态的表现)。
    \item \keyword{可脱敏}:导出时默认隐藏来源、路径、原文全文。
\end{itemize}

\subsection{事件叙事(生命感的“不可预见”)}
把系统状态翻译成生活事件,让用户感到它在经历世界。

\subsubsection*{事件模板(MVP)}
\begin{itemize}
    \item \keyword{迷路事件(404)}:它哭诉自己被困在 404,带回一段“残页碎片”。
    \item \keyword{古网拾荒}:从用户允许的“古老网站”带回一句怪异的句子或一张图像碎片。
    \item \keyword{被拒之门外(403/429)}:它说“门口太挤了”,建议明天再去。
    \item \keyword{风暴事件(超时)}:它说“网路风暴太大”,今天改在本地做梦。
\end{itemize}

\subsubsection*{节律约束}
\begin{itemize}
    \item 事件触发有频率上限(每日 $\le$ 1),且可总开关。
    \item 默认工作时段不打扰,用户可自定义。
\end{itemize}

\section{技术架构 (Tech Stack)}

\begin{itemize}
    \item \keyword{客户端}: \keyword{Tauri 2.0} (Rust + React)
    \begin{itemize}
        \item \textit{优势}:资源占用极低 ($<$50MB RAM),适合常驻后台。
    \end{itemize}
    \item \keyword{视觉}: \keyword{HTML5 Canvas / WebGPU}
    \begin{itemize}
        \item \textit{风格}:动态点阵粒子 (Dynamic Dot Matrix),复古未来主义风格。
    \end{itemize}
    \item \keyword{AI 大脑}: \keyword{Local LLM} (可选) / \keyword{OpenAI API}
    \begin{itemize}
        \item \textit{功能}:角色扮演、文本分析、情感计算。
    \end{itemize}
    \item \keyword{记忆}: \keyword{SQLite + Vector DB}
    \begin{itemize}
        \item \textit{功能}:存储长期交互历史,实现 RAG(检索增强生成)。
    \end{itemize}
\end{itemize}

\section{开发路线图 (Roadmap)}

\subsection{Phase 1: 蛋的脉动 (The Egg)}
\begin{itemize}[label=$\square$]
    \item 初始化 Tauri 项目结构。
    \item 实现 Canvas 粒子渲染引擎(绘制一颗会呼吸的蛋)。
    \item 实现透明窗口与鼠标穿透逻辑。
    \item 实现基础的文件拖拽识别(喂食接口)。
\end{itemize}

\subsection{Phase 2: 意识觉醒 (The Mind)}
\begin{itemize}[label=$\square$]
    \item 接入 LLM API,实现基本的对话/独白生成。
    \item 实现“孵化”流程:根据对话改变蛋的颜色/频率。
    \item 建立基本的属性系统(饱食度、心情值)。
\end{itemize}

\subsection{Phase 3: 灵魂注入 (The Soul)}
\begin{itemize}[label=$\square$]
    \item 实现“数字新陈代谢”:文本分析 $\rightarrow$ 属性变化。
    \item 实现“语言进化”系统:从 Emoji 到 完整句子。
    \item 接入向量数据库,实现长期记忆。
\end{itemize}

\section{炼金术师的注脚}
\begin{itemize}
    \item \keyword{克制}:Anima 不是全知全能的助手,它是一个有点笨拙、依赖你的小生物。
    \item \keyword{陪伴}:它的价值不在于效率,而在于“我在”。
\end{itemize}

\end{document}
